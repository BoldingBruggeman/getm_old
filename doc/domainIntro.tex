\section{Introduction to the calculation domain}

\vspace{0.5cm}

  This module handles all tasks related to the definition of the
  computational domain - except reading in variables from file.
  The required information depends on the $grid\_type$ and also on the
  complexity of the model simulation to be done.\newline
  The mandatory varible $grid\_type$ read from the file containing
  the bathymetry and coordinate information (presently only NetCDF
  is supported) is guiding subsequent tasks. $grid\_type$ can take
  the following values:
  \begin{itemize}
      \item[1:] equi-distant plane grid - $dx$, $dy$ are constant - but
                not necessarily equal
      \item[2:] equi-distant spherical grid  - $dlon$, $dlat$ are
                constant - and again not necessarily  equal
      \item[3:] curvilinear grid in the plane - $dx$, $dy$ are both
                functions of (i,j). The grid must be orthogonal
%      \item[4:] curvilinear grid on a sphere - $dx$, $dy$ are functions
%                of (i,j) and are calculated under the assumption of a
%                perfect sphere
  \end{itemize}

  For all values of $grid\_type$ the bathymetry given on the T-points
  (see the GETM manual for definition) must be given.\newline

  Based on the value of $grid\_type$ the following additional variables
  are required:
  \begin{itemize}
      \item[1:] proper monotone coordinate informtion in the xy-plane
                with equidistant spacing. The name of the coordinate
                variables are $xcord$ and $ycord$.
      \item[2:] proper monotone coordinate informtion on the sphere
                with equidistant spacing in longitude and latitude. The
                names of the coordinate variables are $xcord$ and $ycord$.
      \item[3:] position in the plane of the grid-vertices. These are
                called X-points in GETM. The names of these two variables
                are $xx$ and $yx$.
%      \item[4:] position on the sphere of the grid-vertices. The names
%                of these variables are clled $lonx$ and $latx$
  \end{itemize}

  In addition to the above required grid information the following
  information is necessary for specific model configurations:
  \begin{itemize}
      \item[A:] $latu$ and $latv$ \newline
             If $f\_plane$ is false information about the
             latitude of U- and V-points are required for calculating
             the Coriolis term correctly. For $grid\_type=1$
             $latu$ and $latv$ are calculated based on an additional field
             $latc$ i.e. the latitude of the T-points. For $grid\_type=3$
             $latx$ i.e. the latitude of the X-points will have to be
             provided in order to calculate $latu$ and $latv$.
      \item[B:] $lonc$, $latc$ and $convc$ \newline
             The longitude, latitude positions of
             the T-points are required when using forcing from a NWP-model.
             $lonc$ and $latc$ are used to do spatial interpolation from
             the meteo-grid to the GETM model and $convc$ is the rotation
             of the local grid from true north.
  \end{itemize}

  In addition to the information above a few files are optionally read
  in $init\_domain()$. Information about open boundaries, modifications
  to the bathymetry and the calculation masks are are done via simple
  ASCII files. \newline

\vspace{0.5cm}




