\section{Introduction}

\subsection{What is GETM?}


\subsection{A short history of GETM}\label{SectionHistory}

The idea for GETM was born in May 1997 in Arcachon, France during a 
workshop of the PhaSE project which was sponsored by the European Community
in the framework of the MAST-III programme. It was planned  
to set up an idealised numerical model for the Eastern Scheldt, The Netherlands
for simulating the effect of vertical mixing of nutrients on filter feeder
growth rates. A discussion between the first author of this report,
Peter Herman (NIOO, Yerseke, The Netherlands) and Walter Eifler
(JRC Ispra, Italy) had the result that the
associated processes were inherently three-dimensional (in space), and thus,
only a three-dimensional model could give satisfying answers. 
Now the question arose, which numerical model to use. An old
wadden sea model by \cite{BURCHARD95} including a two-equation turbulence
model was written in $z$-coordinates with fixed geopotential layers
(which could be added or removed for rising and sinking sea surface elevation,
respectively) had proven to be too noisy for the applications in mind.
Furthermore, the step-like bottom approximation typical for such models
did not seem to be sufficient. Other Public Domain models did not allow for
drying and flooding of inter-tidal flats, such as the Princeton Ocean Model
(POM). There was thus the need for a new model. Most of the ingredients were
however already there. The first author of this report had already written a
$k$-$\eps$ turbulence model, see \cite{BURCHARDea95}, the forerunner
of GOTM. A two-dimensional code for general vertical coordinates had been
written as well, see \cite{BURCHARDea97}. And the first author of this report
had already learned a lot about mode splitting models from
Jean-Marie Beckers (University of Liege, Belgium).   
Back from Arcachon in Ispra, Italy at the Joint Research Centre
of the European Community, the model was basically written during six weeks,
after which an idealised tidal simulation for the Sylt-R\o m\o{} Bight
in the wadden sea area between Germany and Denmark could be successfully
simulated, see \cite{BURCHARD98}.
By that time this model had the little attractive name {\it MUDFLAT}
which at least well accounted for
the models ability to dry and flood inter-tidal flats.  
At the end of the PhaSE project in 1999, the 
idealised simulation of mussel growth in the Eastern Scheldt could be
finished (not yet published, pers.\ comm.\ Francois Lamy and Peter Herman).  

In May 1998 the second author of this report joined the development of
{\it MUDFLAT}. He first fully rewrote the model from a one-file FORTRAN77 code
to a modular FORTRAN90/95 code,
made the interface to GOTM (such that the original $k$-$\eps$ model
was not used any more), integrated the netCDF-library into the model,
and prepared the parallelisation of the model. 
And a new name was created, GETM, General Estuarine Transport Model. 
As already in GOTM, the word "General" does not imply that the model
is general, but indicates the motivation to make it more and more general. 

At that time, GETM has actually been applied for simulating currents inside
the Mururoa atoll in the Pacific Ocean, see \cite{MATHIEUea01}. 

During the year 2001, GETM was then extended by the authors
of this report to be a fully baroclinic
model with transport of active and passive tracers, calculation of
density, internal pressure gradient and stratification, surface heat and
momentum fluxes and so forth. During a stay of the first author
at the Universit\'e Catholique de Louvain,
Institut d'As\-tro\-no\-mie et de G\'eophysique George Lema\^\i tre,
Belgium (we are grateful to Eric Deleersnijder for this
invitation and many discussions) the high-order advection schemes have
been written. 
During another invitation to Belgium, this time to the
GHER at the Universit\'e de Li\`ege, the first author had the
opportunity to discuss numerical details of GETM with Jean-Marie
Beckers, who originally motivated us to use the mode splitting technique. 

The typical challenging application
in mind of the authors was always a simulation of the tidal Elbe,
where baroclinicity and drying and flooding of inter-tidal flats play
an important role. Furthermore, the tidal Elbe is long, narrow  and bended,
such that the use of Cartesian coordinates would require an indexing
of the horizontal fields, see e.g.\ \cite{DUWE88}. Thus, the use of
curvi-linear coordinates which follow the course of the river
has already been considered for a long time.   
However, the extensions just listed above, give the model also the
ability to simulate shelf sea processes in fully baroclinic mode,
such that the name General Estuarine Transport Model is already 
a bit too restrictive. 
