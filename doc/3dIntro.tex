%$Id: 3dIntro.tex,v 1.2 2007-01-10 22:01:16 hb Exp $

\section{Introduction to 3d module}

\subsection{Overview over 3D routines in GETM}
\label{Section_Overview_3D}


This module contains the physical core of GETM. All three-dimensional
equations are iterated here, which are currently the equations for

\vspace{0.5cm}

\begin{tabular}{llllll}
quantity & description & unit & variable & routine name & page \\
$p_k$    & layer-int.\ $u$-transport & m$^2$s$^{-1}$ & {\tt uu} & 
{\tt uu\_momentum} & \pageref{sec-uu-momentum-3d} \\
$q_k$    & layer-int.\ $v$-transport & m$^2$s$^{-1}$ & {\tt vv} & 
{\tt vv\_momentum} & \pageref{sec-vv-momentum-3d} \\
$\theta$    & potential temperature & $^{\circ}$C & {\tt T} & 
{\tt do\_temperature} & \pageref{sec-do-temperature} \\
$S$    & salinity & psu & {\tt S} & 
{\tt do\_salinity} & \pageref{sec-do-salinity} \\
$C$    & suspended matter & kg\,m$^{-3}$ & {\tt spm} & 
{\tt do\_spm} & \pageref{sec-do-spm} \\
\end{tabular}

\vspace{0.5cm}

The vertical grid for GETM, i.e.\ the layer thicknesses in all
U-, V- and T-points, are defined in the routine {\tt coordinates},
see section \ref{sec-coordinates} on page \ref{sec-coordinates}.

The grid-related vertical velocity $\bar w_k$ is calculated directly from 
the layer-integrated continuity equation (\ref{ContiLayerInt}) which here done
in the routine {\tt ww\_momentum} described on page \pageref{sec-ww-momentum-3d}.

The physics of the horizontal momentum equations is given in
section \ref{Section_3d_momentum}, and their transformation to general vertical
coordinates in section \ref{SectionLayerIntegrated}. Their numerical
treatment will be discussed in the routines for the individual terms, see
below.
The forcing terms of the horizontal momentum equations are calculated in 
various routines, such as {\tt uv\_advect\_3d} for the three-dimensional 
advection (which in turn calls 
{\tt advection\_3d} in case that higher order positive definite advection
schemes are chosen for the momentum equation), 
{\tt uv\_diffusion\_3d.F90} for the horizontal diffusion, 
{\tt bottom\_friction\_3d} for the bottom friction applied to the lowest
layer, and {\tt internal\_pressure} for the calculation of the internal
pressure gradients. 

The major tracer equations in any ocean model are those for potential
temperature and salinity. They are calculated in the routines 
{\tt do\_temperature} and {\tt do\_salinity}. A further hard-coded 
tracer equation is the suspended matter equation, see {\tt do\_spm}.

In the near future (the present text is typed in February 2006),
a general interface to the biogeochemical module of GOTM 
(also not yet released) will be available. 
This allow to add tracer equations of arbitrary complexity to GETM, ranging
from completely passive tracer equations to complex ecosystem models
such as ERSEM (\cite{BARETTAea95}). The interfacing between this
so-called GOTM-BIO to GETM is made in a similar manner than the interfacing
between GETM  and the GOTM turbulence module described in 
{\tt gotm} on page \pageref{sec-gotm}.
The basic structure of GOTM-BIO has been recently presented by 
\cite{BURCHARDea06}.
Some more details about the tracer equations currently included in
GETM is given in section \ref{Section_tracer}.

The entire turbulence model, which basically provides eddy viscosity
$\nu_t$ and eddy diffusivity $\nu'_t$ is provided from the General
Ocean Turbulence Model (GOTM, see \cite{UMLAUFea05} for the
source code documentation and {\tt http://www.gotm.net} download of
source code, docomentation and test scenarios). The turbulence module
of GOTM (which is a complete one-dimensional water column model) is coupled
to GETM via the interfacing routine {\tt gotm} described in section
{\tt gotm} on page \pageref{sec-gotm}. Major input to the turbulence model
are the shear squared 
$M^2=\left(\partial_zu\right)^2+\left(\partial_zu\right)^2$ and the buoyancy
frequency squared $N^2=\partial_z b$ with the buoyancy $b$ from (\ref{bdef}).
Those are calculated and interpolated to the T-points where the 
turbulence model columns are located in the routine {\tt ss\_nn}
described on page \pageref{sec-ss-nn}.

The surface and bottom stresses which need to be passed to the turbulence
module as well, are interpolated to T-points in the routine
{\tt stresses\_3d}, see page \pageref{sec-stresses-3d}.

The module {\tt rivers} (see section \ref{sec-rivers} on page 
\pageref{sec-rivers}) organises the riverine input of fresh water
from any number of rivers.

Three-dimensional boundary conditions for temperature and salinity
are provided by means of the module {\tt bdy-3d}, see section
\ref{bdy-3d} described on page \pageref{bdy-3d}.

The remaining routines in the module {\tt 3d} deal with the coupling of the
external and the internal mode. The basic idea of the mode splitting
has already been discussed in section \ref{Section_mode_splitting}.
The consistency of the two modes is given through the so-called
slow terms, which are mode interaction terms resulting from
subtracting vertically integrated equations with parameterised 
advection, diffusion, bottom friction and internal pressure 
gradient from vertically integrated equations with explicit vertical resolution
of these processes. These slow terms which are updated every macro
time step only (that is why we call them slow terms) need to be considered
for the external mode included in module {\tt 2d}.
Those slow terms are calculated here in the {\tt 3d} module in the routines
{\tt slow\_bottom\_friction}, {\tt slow\_advection} 
and {\tt slow\_diffusion}, and they are added together in {\tt slow\_terms},
see the descriptions in sections \ref{sec-slow-bottom-friction}, 
\ref{sec-slow-advection},\ref{sec-slow-diffusion} and \ref{sec-slow-terms}
on pages \pageref{sec-slow-bottom-friction}, 
\pageref{sec-slow-advection},\pageref{sec-slow-diffusion} and 
\pageref{sec-slow-terms},
respectively.

One other important measure of coupling the two modes is to
add to all calculated $u$- and $v$-velocity profiles the difference 
between their vertical integral and the time-average of the 
vertically integrated transport from the previous set of micro time 
steps. This shifting is done in the routines {\tt uu\_momentum\_3d} and
{\tt vv\_momentum\_3d} and the  time-average of the
vertically integrated transport is updated in the {\tt 2d} module 
in the routine {\tt m2d} and divided by the number of
micro time steps per macro time step in {\tt start\_macro}. 
Further basic calculations performed in {\tt start\_macro}
(see description in section \ref{sec-start-macro} 
on page \pageref{sec-start-macro}) are the updates of
the {\it old} and {\it new} sea surface elevations with respect to 
the actual macro time step.
The routine {\tt stop\_macro} (see description in section \ref{sec-stop-macro}
on page \pageref{sec-stop-macro}) 
which called at the end of each macro time step
simply resets the variables for the time-averaged transports to zero. 

\subsection{Tracer equations}\label{Section_tracer}

The general conservation equation for tracers $c^i$ with $1\leq i \leq N_c$
(with $N_c$ being the number of tracers), 
which can e.g.\ be temperature,
salinity, nutrients, phytoplankton, zoo-plankton, suspended matter,
chemical concentrations etc.\ is given as:

\begin{equation}\label{densz}
\begin{array}{l}
\partial_t c^i +\partial_x (uc^i) +\partial_y(vc^i) 
+\partial_z ((w+\alpha w_s^i)c^i)
-\partial_z(\nu_t' \partial_z c^i)
\\ \\ \displaystyle \qquad
-\partial_x(A_h^T \partial_x c^i)
-\partial_y(A_h^T \partial_y c^i)
=Q^i.
\end{array} 
\end{equation}

Here, $\nu'_t$ denotes the vertical eddy diffusivity and
$A_h^T$ the horizontal diffusivity.
Vertical migration of concentration with migration velocity
$w_s^i$ (positive for upward motion) is considered as well. This could be 
i.e.\ settling of suspended matter or active migration of
phytoplankton.  
In order to avoid stability problems with vertical advection when 
intertidal flats are drying, the settling of SPM is linearly 
reduced towards zero when the water
depth is between the critical and the minimum water depth. 
This is
done by means of multiplication of the settling velocity with $\alpha$,
(see the definition in equation (\ref{alpha})).
$Q^i$ denotes all internal sources and sinks of the tracer $c^i$.
This might e.g.\ be for the temperature equation the heating of water
due to absorption of solar radiation in the water column.

Surface of bottom boundary conditions for tracers are usually given
by prescribed fluxes:

\begin{equation}\label{SurfFlux}
\nu_t' \partial_z c^i = F^i_s  \qquad \mbox{for } z=\zeta
\end{equation}

and 

\begin{equation}\label{BottFlux}
\nu_t' \partial_z c^i = -F^i_b  \qquad \mbox{for } z=-H,
\end{equation}

with surface and bottom fluxes $F^n_s$ and $F^n_b$ directed
into the domain, respectively. 

At open lateral boundaries, the tracers $c^n$ are prescribed for
the horizontal velocity normal to the open boundary 
flowing into the domain. In case of outflow, a zero-gradient condition is
used. 

All tracer equations except those for temperature, salinity and 
suspended matter will be treated in the future by means of GOTM-BIO.

The two most important tracer equations which are hard-coded in GETM
are the transport equations for potential temperature $\theta$ in $^{\circ}$C
and salinity $S$ in psu (practical salinity units):

\begin{equation}\label{Teq}
\begin{array}{l}
\partial_t \theta +\partial_x (u\theta) +\partial_y(v\theta) +\partial_z (w\theta)
-\partial_z(\nu_t' \partial_z \theta)
\\ \\ \displaystyle \qquad
-\partial_x(A_h^\theta \partial_x \theta)
-\partial_y(A_h^\theta \partial_y \theta)
=\frac{\partial_z I}{c'_p \rho_0},
\end{array} 
\end{equation}

\begin{equation}\label{Seq}
\begin{array}{l}
\partial_t S +\partial_x (uS) +\partial_y(vS) +\partial_z (wS)
-\partial_z(\nu_t' \partial_z S)
\\ \\ \displaystyle \qquad
-\partial_x(A_h^S \partial_x S)
-\partial_y(A_h^S \partial_y S)
=0.
\end{array} 
\end{equation}

On the right hand side of the temperature equation (\ref{Teq})
is a source term 
for absorption of solar radiation with the solar radiation at depth $z$,
$I$, and the specific heat capacity of water, $c'_p$. 
According to \cite{PAULSONea77} the radiation $I$ in the upper
water column may be parameterised by

\begin{equation}\label{Light}
I(z) = I_0 \left(ae^{-\eta_1z}+(1-a)e^{-\eta_2z}\right).
\end{equation}

Here, $I_0$ is the albedo corrected radiation normal to the sea surface.
The weighting parameter $a$ and the 
attenuation lengths for the
longer and the shorter fraction of the short-wave radiation,
$\eta_1$ and $\eta_2$, respectively, depend on the turbidity of the water.
\cite{JERLOV68} defined 6 different classes of water
from which \cite{PAULSONea77} calculated weighting parameter $a$
and attenuation coefficients $\eta_1$ and $\eta_2$.

At the surface, flux boundary conditions for $T$ and $S$ have to
be prescribed. For the potential temperature, it is of the following form:  

\begin{equation}\label{TempFlux}
\nu'_t \partial_z T= \frac{Q_s+Q_l+Q_b}{c'_p \rho_0},
\qquad \mbox{for } z=\zeta,
\end{equation}

with the sensible 
heat flux, $Q_s$, the 
latent heat flux, $Q_l$ and the
long wave back radiation,
$Q_b$. Here, the \cite{KONDO75} 
bulk formulae have been used for calculating the
momentum and temperature surface fluxes
due to air-sea interactions. 
In the presence of sea ice, these air-sea fluxes have to be considerably
changed, see e.g.\ \cite{KANTHAea00b}. 
Since there is no sea-ice model coupled to GETM presently,
the surface heat flux is limited to positive values, 
when the sea surface temperature $T_s$ reaches the freezing point

\begin{equation}\label{freezingpoint}
T_f=-0.0575\,S_s+1.710523\cdot 10^{-3}\, S_s^{1.5}
-2.154996\cdot 10^{-4}\,S_s^2\approx -0.0575\,S_s
\end{equation}

with the sea surface salinity $S_s$, see e.g.\ \cite{KANTHAea00}:

\begin{equation}\label{primitive_ice_model} 
Q_{surf} = \left\{
\begin{array}{ll}
Q_s+Q_l+Q_b, & \mbox{ for } T_s > T_f, \\ \\
\max\{0,Q_s+Q_l+Q_b\}, & \mbox{ else.}
\end{array}
\right.
\end{equation}

For the surface freshwater flux,
which defines the salinity flux, the difference between evaporation $Q_E$
(from 
bulk formulae) and precipitation $Q_P$ (from observations or
atmospheric models) is calculated:

\begin{equation}\label{SalFlux}
\nu'_t\partial_z S = \frac{S(Q_E-Q_P)}{\rho_0(0)},
\qquad \mbox{for } z=\zeta,
\end{equation}

where $\rho_0(0)$ is the density of freshwater at sea surface temperature.
In the presence of sea-ice, the calculation of freshwater flux
is more complex, see e.g.\ \cite{LARGEea94}.
However, for many short term calculations, the freshwater
flux can often be neglected compared to the surface 
heat flux.

A complete revision of the surface flux calculation is currently under
development. It will be the idea to have the same surface flux calculations
for GOTM and GETM. In addition to the older bulk formulae by \cite{KONDO75}
we will also implement the more recent formlations by 
\cite{FAIRALLea96}.

Heat and salinity fluxes at the bottom are set to zero. 

\subsection{Equation of state}\label{Section_state_eq}

The coupling between the potential temperature and salinity 
equations and the momentum
equations is due to an algebraic equation of state:

\begin{equation}\label{UNESCO} 
\rho=\rho(\theta,S,p_0)
\end{equation}

with $p_0=g\rho_0(\zeta-z)$ being the hydrostatic reference pressure.
In order to obtain potential density from the equation of state, 
$p_0$ needs to be set to zero, which is the default in GETM.

Currently the equation of state by \cite{FOFONOFFea83} is implemented
into GETM, but the more recent and more consistent equation of state
by \cite{JACKETTea05} which is already contained in GOTM
will be added as an option in the near future.

For the equation of state, also linearised version are implemented into
GETM, for details, see section \ref{sec-eqstate} on page \pageref{sec-eqstate}.

It should be noted carefully that the GETM variable {\tt rho} actually
stands for the buoyancy $b$ as defined in (\ref{bdef}).
